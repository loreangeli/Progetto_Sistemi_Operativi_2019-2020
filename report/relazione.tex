\documentclass{article} %tipo del documento
\usepackage{fontspec} %font times new roman
\usepackage{listings} %pacchetto per il codice
\lstloadlanguages{C}
\setmainfont{Times New Roman}
\usepackage{setspace} %interlinea singola
\singlespacing 
\usepackage[a4paper,top=2cm,bottom=3cm,left=2cm,right=2cm, heightrounded, bindingoffset=0mm]{geometry} %margini 2 cm

\title{Relazione Progetto Sistemi Operativi}
\author{Lorenzo Angeli - Corso A - 539036}
\date{2019/2020}
\usepackage{graphicx}

\begin{document}

\maketitle

\section{Supermercato}
L’architettura del supermercato è costituita da tre componenti (o thread) principali: Cliente, Cassiere e Direttore. 
\subsection{Cliente}
Il cliente viene creato in modalità detached dal main ({\itshape supermercato.c}). Il meccanismo di ricerca della cassa aperta dove potersi mettere in coda per pagare, avviene in modo randomico. I clienti con 0 prodotti acquistati non vengono inseriti nella coda ma attendono il permesso dal direttore per poter uscire dal supermercato (vedi paragrafo 1.3.2).
\begin{lstlisting}
elemcassa* lista;
static pthread_mutex_t locklista=PTHREAD_MUTEX_INITIALIZER; 
\end{lstlisting}
Ho utilizzato una lista {\itshape lista} di interi (presente nella libreria {\itshape dati} ) dove poter salvare gli indici delle casse aperte utilizzando una variabile di mutex {\itshape locklista} per eseguire le varie operazioni in mutua esclusione. La ricerca di una cassa aperta avviene in questo modo:
 \begin{enumerate} 
 \item Salvo all'interno di una variabile {\itshape dimlista} il numero degli elementi presenti all'interno della lista.
 \item Genero un valore casuale {\itshape tmp} compreso tra [0, {\itshape dimlista}-1]
 \item Cerco l'elemento che corrisponde all'indice {\itshape tmp} nella lista, esso corrisponderà alla cassa in cui andrò ad inserire il cliente.
 \end{enumerate}
Terminato l’inserimento in coda alla cassa, il thread cliente viene fatto terminare. A questo punto sarà il cassiere ad occuparsi della gestione del cliente. 
\subsection{Cassiere}
Il cassiere è stato implementato con attesa tramite join per favorire la gestione di terminazione del programma. Ogni cassiere ha la propria lock e variabile di condizione. 
\begin{lstlisting}
pthread_mutex_t* mutex;
pthread_cond_t* cond;
\end{lstlisting}
La variabile {\itshape mutex[i]} serve per avere mutua esclusione sulla modifica delle strutture dati associate alla cassa i, mentre {\itshape cond} è una variabile di condizione che ha lo scopo di evitare attesa attiva nel ciclo while che controlla lo stato della cassa e la presenza di clienti in coda. Infatti, nel caso di cassa chiusa o di cassa aperta senza clienti in coda, il thread cassiere i viene fatto sostare nella coda della variabile di condizione  {\itshape cond} per poi essere riattivato dal cliente tramite signal. Terminata l’operazione di rimozione del cliente dalla cassa tramite funzione {\itshape rimuovi} della libreria {\itshape dati} può essere servito il cliente. Vediamo adesso quali sono le strutture dati più importanti utilizzate dal Cassiere di cui fanno uso anche le funzioni della libreria {\itshape dati}:
\begin{enumerate} 
 \item int* codaclienti: array di K posizioni in cui salvo il numero di clienti in coda, dove ogni posizione dell'array corrisponde ad una cassa ben precisa.
 \item int* statocassa: array di K posizioni che indica se la cassa i è aperta (vale 1) o chiusa (vale 0).
 \end{enumerate}
\subsection{Direttore}
Il direttore è unico e principalmente si occupa di mantenere consistenti le informazioni sulle casse e di gestire i clienti che vogliono uscire dal supermercato senza aver acquistato nessun prodotto. La politica di apertura/chiusura casse rispetta i requisiti del progetto. Utilizza però una piccola variazione per evitare che il direttore causi una apertura di una cassa seguita dall'immediata chiusura della stessa o di un altra cassa. E' quindi possibile aprire sempre nuove casse (se le condizioni lo permettono). Se però nell'ultimo istante è stata chiusa una cassa allora non è possibile aprirne una nuova.
\subsubsection{Comunicazione tra Direttore e Cassieri}
Vediamo di seguito il codice utilizzato per gestire la comunicazione dei cassieri nei confronti del direttore per informarlo sul numero dei clienti in coda alle casse:
\begin{lstlisting}
pthread_mutex_t locknotify=PTHREAD_MUTEX_INITIALIZER;
pthread_cond_t attendinotify=PTHREAD_COND_INITIALIZER; 
//array di bit (0/1)
int* notifica;  
int* cassenotify; 
int* clientinotify; 
\end{lstlisting}
Per la comunicazione tra cassieri e direttore si utilizzano K thread di supporto creati in modalità detached, uno per cassiere. Ognuno di questo thread i scrive le proprie informazioni riguardanti la cassa corrispondente dentro due array di interi ({\itshape cassenotify} e {\itshape clientinotify}) ad intervalli regolari e settano a 1 {\itshape notifica[i]} che ha il compito di capire quale cassa deve ancora inviare le proprie informazioni. {\itshape cassenotify} contiene le informazioni sugli stati delle casse mentre {\itshape clientinotify} contiene le informazioni sul numero di clienti in coda alle casse. Il direttore legge le informazioni da questi due array quando ognuno dei cassieri ha comunicato le proprie informazioni e se ne accorge quando {\itshape notifica} ha ogni cella settata a 1. A questo punto il direttore legge i dati e li salva dentro due array di supporto. Per mantenere la mutua esclusione di queste strutture viene utilizzata una lock {\itshape locknotify} e una variabile di condizione {\itshape attendinotify} con lo scopo di evitare l’attesa attiva da parte del direttore nel ciclo while che controlla se ogni cassiere ha inviato i propri dati. 
\subsubsection{Gestione uscita dei clienti con 0 prodotti}
Vediamo di seguito il codice utilizzato per gestire l'uscita dei clienti senza acquisti e successivamente commentiamolo.
\begin{lstlisting}
int permesso; 
pthread_mutex_t lockuscita=PTHREAD_MUTEX_INITIALIZER; 
pthread_cond_t condcliente=PTHREAD_COND_INITIALIZER; 
\end{lstlisting}
Per gestire i clienti che devono chiedere l’autorizzazione al direttore per poter uscire che non hanno acquistato nessun prodotto si utilizzano: una variabile intera (flag) {\itshape permesso} (che vale 1 se il direttore ha dato il via libera ai clienti di uscire e 0 altrimenti), una lock {\itshape lockuscita} e una variabile di condizione {\itshape condcliente}. {\itshape lockuscita} serve per mantenere la mutua esclusione della variabile {\itshape permesso} mentre nella coda della variabile di condizione {\itshape condcliente} si bloccano i thread clienti con 0 acquisti in attesa del permesso da parte del direttore di poter uscire. 

\section{Librerie}
Tutte le librerie sono memorizzate nella directory {\itshape lib}. Descriviamo di seguito le librerie utilizzate dalle componenti principali.
\subsection{dati}La libreria  {\itshape dati} contiene due strutture dati: un array di liste {\itshape coda} e una semplice linked list {\itshape lista}. Ogni posizione dell'array {\itshape coda} fa riferimento a una cassa e ogni cassa contiene la lista dei clienti in coda. La lista {\itshape lista} invece contiene l'indice delle casse aperte.
Le principali funzioni di {\itshape coda} sono:
\begin{itemize}
\item inserisci: inserisce il cliente in fondo alla lista di una delle casse aperte. L’inserimento non avviene in maniera casuale ma deve essere specificata la cassa dove voler inserire il dato. 
\item rimuovi: rimuove il primo elemento inserito dalla cassa specificata.
\item opencassa: apre la cassa passata come argomento se risulta chiusa. 
\item closecassa: chiude la cassa passata come argomento se aperta. La chiusura di una cassa sposta ogni elemento della cassa chiusa in una delle casse aperte.
\end{itemize} 
Le principali funzioni di {\itshape lista} sono:
\begin{itemize}
\item inseriscicassa:inserisce l'indice della cassa passata come argomento in coda alla lista.
\item rimuovicassa: rimuove l'indice della cassa passata come argomento dalla lista.
\item generacassa: restituisce casualmente l'indice di una delle casse presenti all'interno della lista.
\end{itemize}
\subsection{parsing} La libreria parsing serve per effettuare la lettura del file {\itshape config.txt} ed estrapolarne le informazioni in modo che possano essere salvate in alcune variabili globali all'interno del main.
 \subsection{random} La libreria random contiene le funzioni che servono per generare valori casuali utilizzate dai thread principali.
 
\section{Ulteriori Dettagli Implementativi}
\subsection{Gestione dei Segnali}
Per gestire i segnali SIGQUIT e SIGHUP sono stati utilizzati i seguenti flag globali:

\begin{lstlisting}
volatile sig_atomic_t segnalesq; //SIGQUIT
volatile sig_atomic_t segnalesh; //SIGHUP
\end{lstlisting}
dove volatile sig\_atomic\_t è un tipo intero a cui si può accedere atomicamente e che quindi si può usare all'interno di un signal-handler. Ogni altro accesso a variabile globale (non dichiarata volatile sig\_atomic\_t) può avere un comportamento non definito.
Inizialmente vengono settati a 0 nel main, ma nel momento in cui viene generato un segnale, tramite la propria funzione signal-handler impostano la propria variabile a 1. Immediatamente prima di uscire dalla funzione si tenta di risvegliare eventuali cassieri bloccati nelle variabili di condizione {\itshape cond[i]} per evitare che si verifichino deadlock. 
\subsection{Conteggio clienti attivi}
Per rimanere informati sui clienti (thread) attivi nel supermercato si fa riferimento al seguente codice:
\begin{lstlisting}
int nclienti;
pthread_mutex_t lockclienti=PTHREAD_MUTEX_INITIALIZER;  
\end{lstlisting}
La variabile {\itshape nclienti} contiene il numero dei clienti attivi nel supermercato. Per mantenere aggiornata tale variabile essa viene incrementata all'ingresso di ogni cliente nel supermercato e viene decrementata dal cassiere dopo che questo è stato servito. A tal proposito poichè più thread modificano il valore di questa variabile è necessario introdurre una lock {\itshape lockclienti} in modo da poterla utilizzare in mutua esclusione. Parallelamente a questo codice sono state implementate tre funzioni in modo da poter eseguire le modifiche alla variabile {\itshape nclienti} in maniera pulita:
\begin{itemize}
 \item void inc(int* nclienti); //incrementa {\itshape nclienti} in mutua esclusione. 
 \item void dec(int* nclienti); //decrementa {\itshape nclienti} in mutua esclusione.
 \item int get(int* nclienti); //restituisce il valore di {\itshape nclienti}.
 \end{itemize} 
 \subsection{Scrittura statistiche su file}
 La scrittura delle statistiche riguardanti il Direttore, i Clienti e i Cassieri vengono eseguite sul file {\itshape filelog.txt}. Per poter eseguire tale scrittura è stato necessario utilizzare una variabile di mutua esclusione {\itshape filemtx}. Questo perché più thread clienti potrebbero scrivere sul file nello stesso istante. 
\section{Compilazione e Debugging}
\subsection{Compilazione}
Per compilare il codice del programma è sufficiente scompattare l'archivio {\itshape lorenzo\_angeli-corsoA.tar.gz}, spostarsi in tale directory ed eseguire il comando {\itshape make}. Vengono messi a disposizione altri comandi come {\itshape make test2} utile per eseguire un test del programma {\itshape supermercato} con invio di un segnale SIGHUP dopo 25s, il comando {\itshape make test1} simile al test2 ma con la differenza che questo invia un segnale SIGQUIT, il comando {\itshape make clean} per ripulire gli eventuali file creati in precedenza con l'esecuzione di {\itshape make} ed infine il comando {\itshape make all} che consente di eseguire il programma {\itshape supermercato}. Attenzione però perché in questo caso il programma non riceve nessun segnale SIGHUP/SIGQUIT in automatico ma è necessario inviare rispettivamente i comandi {\itshape killall -1 supermercato} o {\itshape killall -3 supermercato}.
\subsection{Debugging}
Per eseguire il debugging è sufficiente modificare i flag che si trovano nell'intestazione del file {\itshape supermercato.c} dal valore 0 al valore 1. Ogni flag fa riferimento a una parte del codice diversa che può essere debuggata.
\begin{lstlisting}
#define DEBUGC 0 //DEBUG cliente
#define DEBUGCA 0 //DEBUG cassiere
#define DEBUGD 0 //DEBUG direttore
#define DEBUGM 0 //DEBUG main
\end{lstlisting}

\end{document}
